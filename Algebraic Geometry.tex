% chktex-file 1 chktex-file 8 chktex-file 9 chktex-file 12 chktex-file 13 chktex-file 15 chktex-file 17 chktex-file 18 chktex-file 26 chktex-file 31 chktex-file 36 chktex-file 44
\documentclass[10pt]{article}
\input{/Users/mr.hassium/Desktop/Github/Hello-World/hassium.tex}
\begin{document}
\def\htitle{Algebraic Geometry}
\def\hauthor{Hassium}
\hsetup
\htoc
\hmain
\section{Varieties}
\begin{definition}
    Let $K$ be an algebraically closed field. The \hdef{affine n-space} ${\bb{A}}_{K}^{n}$ is a set $\{({a}_{1},\dots,{a}_{n})\mid{a}_{i}\in K\}$. An element $P\in{\bb{A}}_{K}^{n}$ is called a \hdef{point}, if $P=({a}_{1},\dots,{a}_{n})$, each ${a}_{i}$ is called a \hdef{coordinate} of $P$.
\end{definition}
\begin{remark}
    We will write $\A{n}$ for ${\bb{A}}_{K}^{n}$.
\end{remark}
\par
Let $A=K[{x}_{1},\dots,{x}_{n}]$ be a polynomial ring, then $A$ can be expressed as a function such that for all $f\in A$ and $P=({a}_{1},\dots,{a}_{n})\in\A{n}$, $f(P)=f({a}_{1},\dots,{a}_{n})$, which substitues ${x}_{i}$ by ${a}_{i}$. 
\begin{definition}
    Let $K$ be an algebraically closed field and $A=K[{x}_{1},\dots,{x}_{n}]$. Let $T\sub A$, the \hdef{zero set} of $T$ is the set $Z(T)=\{P\in\A{n}\mid f\in T\ \text{and}\ f(P)=0\}$.
\end{definition}
\par
Let $T\sub A$ and let $J$ be the ideal generated by $T$. For all $P\in Z(J)$ and $f\in T\sub J$, $f(P)=0$, so $Z(J)\sub Z(T)$. For all $f={\sum}_{1}^{k}{a}_{i}{t}_{i}$, where ${a}_{i}\in A$ and ${t}_{i}\in T$, and $P\in Z(T)$, we have ${t}_{i}(P)=0$, so $f(P)=0$, hence $Z(T)=Z(J)$. Since $K$ is a field, $K$ is a PID, so $K$ is noetherian. By Hilbert basis theorem, $A$ is noetherian, then all idea are finitely generated. Let an ideal $J=({f}_{1},\dots,{f}_{r})$, since $Z(T)=Z(J)$, $Z(T)$ is the set of common zeros of those polynomials.
\begin{definition}
    A subset $Y$ of $\A{n}$ is an \hdef{algebraic set} if there exists a subset $T\sub A$ such that $Y=Z(T)$.
\end{definition}
\begin{proposition}
    The union of two algebraic sets is an algebraic set. The intersection of any family of algebraic sets is an algebraic set. The empty set and the whole space are algebraic sets.
\end{proposition}
\begin{proof}
    Let ${\{{Y}_{i}\}}_{i\in I}$ be an arbitrary family of algebraic sets with ${Y}_{i}=Z({T}_{i})$. 
    (\rom1) Consider ${Y}_{1}\cup{Y}_{2}$. For all $P\in{Y}_{1}$, $f\in{T}_{1}$, and $g\in{T}_{2}$, then $fg(P)=f(P)g(P)=0$, so $P\in Z({T}_{1}{T}_{2})$. For all ${P}^{'}\in Z({T}_{1}{T}_{2})$, $fg({P}^{'})=f({P}^{'})g({P}^{'})$, since $A$ is an integral domain, either $f({P}^{'})$ or $g({P}^{'})$ is 0, then ${P}^{'}\in Z({T}_{1})\cup Z({T}_{2})={Y}_{1}\cup{Y}_{2}$. (\rom2) Consider $\In{Y}_{i}=\In Z({T}_{i})$. For all $P\in Z({T}_{i})$ and ${f}_{i}\in{T}_{i}$, ${f}_{i}(P)=0$, then $P\in Z(\Un{T}_{i})$. For all ${P}^{'}\in Z(\Un{T}_{i})$, ${f}_{i}(P)=0$, so $P\in Z({T}_{i})$ for all $i\in I$, which implies $P\in\In Z({T}_{i})=\In{Y}_{i}$. (\rom3) Let $T=(1)$, then $Z(T)=\es$. Let $T=\{0\}$, then $Z(T)=\A{n}$.
\end{proof}
\begin{definition}
    The open subsets of the \hdef{Zariski topology} on $\A{n}$ is the complements of the algebraic sets.
\end{definition}
\par
By the previous proposition, it is trivial that this defines a topology.
\begin{example}
    The open sets of the Zariski topology on $\A{1}$ are the empty set and the complements of finite subsets. This topology is not Hausdorff.
\end{example}
\begin{proof}
    The space $\A{1}=K[x]$. Since $K$ is a field, $K[x]$ is a PID. Since $K$ is algebraically closed, any polynomial can be factorized as $f(x)=c({x}_{1}-{a}_{1})\cdots({x}_{n}-{a}_{n})$, where $c,{a}_{i}\in K$, then $Z(f)=\{{a}_{1},\dots,{a}_{n}\}$. Hence all closed subsets are either finite or $\A{1}$, which is $\A{1}\sm\es$. Suppose the space is Hausdorff. For all $x,y\in\A{1}$, let a desired ${U}_{x}=\A{1}/\{{a}_{1},\dots,{a}_{n}\}\ne\es$, then ${U}_{y}=\{{a}_{1},\dots,{a}_{n}\}$, which is finite, yet contradiction.
\end{proof}
\begin{definition}
    A nonempty subset $Y$ of a topological space $X$ is \hdef{irreducible} if it cannot be expressed as the union $Y={Y}_{1}\cup{Y}_{2}$ of two proper subsets, each one of which is closed in $Y$, the subspace topology.
\end{definition}
\begin{remark}
    The empty set is not considered to be irreducible.
\end{remark}
\begin{example}
    The space $\A{1}$ is irreducible.
\end{example}
\begin{proof}
    All proper closed sets of $\A{1}$ is finite. Since $\A{1}$ is infinite, $\A{1}$ cannot be the union of two proper closed subsets, yet contradiction.
\end{proof}
\begin{example}
    Any nonempty open subset of an irreducible space is irreducible and dense.
\end{example}



\begin{proof}
    Let $X$ be an irreducible space with $S\sub X$ and $S\ne\es$. Consider 
\end{proof}
\begin{example}
    If $Y$ is an irreducible subset of $X$, then its closure $\wb{Y}$ in $X$ is also irreducible.
\end{example}
\begin{proof}
    
\end{proof}
\begin{definition}
    An \hdef{affine algebraic variety} is an irreducible closed subset of $\A{n}$. An open subset of an affine variety is called a \hdef{quasi-affine variety}.
\end{definition}
\begin{remark}
    
\end{remark}
\par




\begin{theorem}[Hilbert's Nullstellensatz]\index{Hilbert's Nullstellensatz}
    Let $K$ be an algebraically closed field, let $J$ be an ideal in the polynomial ring $A=K[{x}_{1},\dots,{x}_{n}]$, and let $f\in A$ vanishes at all points of $Z(J)$. Then ${f}^{r}\in J$ for some integer $r>0$.
\end{theorem}
\begin{proof}
    Consider the \hdef{Rabinowitsch trick}. 
\end{proof}








\begin{proposition}
    If ${T}_{1}\sub{T}_{2}$ are subsets of $A$, then $Z({T}_{2})\sub Z({T}_{1})$. If ${Y}_{1}\sub{Y}_{2}$ are subsets of $\A{1}$, then $I({Y}_{2})\sub I({Y}_{2})$. For any subsets ${Y}_{1},{Y}_{2}\sub\A{n}$, $I({Y}_{1}\cup{Y}_{2})=I({Y}_{1})\cap I({Y}_{2})$. 
\end{proposition}



\pagebreak
\section{Schemes}
\section{Cohomology}
\section{Curves}
\section{Surfaces}
\pagebreak
\newsection{Exercises and Proofs}

\hindex
\end{document}